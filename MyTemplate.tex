\documentclass[UTF8]{ctexart}
\author{张三 19521025}
\date{}
\title{算法设计与分析 作业格式模板}
\pagestyle{plain}
\usepackage{ctex}
\usepackage{fancyhdr}
\usepackage[a4paper]{geometry}
\usepackage{multicol}
\usepackage{amsfonts}
\usepackage{amsmath}
\usepackage{listings}
\usepackage{graphicx}
\usepackage{amssymb}
\usepackage{algorithm}
\usepackage{algorithmicx}
\usepackage{algpseudocode}
\usepackage{float}
\usepackage{lipsum}

\lstset{
    basicstyle          =   \ttfamily,          % 基本代码风格
    keywordstyle        =   \ttfamily,          % 关键字风格
    commentstyle        =   \ttfamily\itshape,  % 注释的风格,斜体
    stringstyle         =   \ttfamily,  % 字符串风格
    flexiblecolumns,                % 别问为什么,加上这个
    numbers             =   left,   % 行号的位置在左边
    showspaces          =   false,  % 是否显示空格,显示了有点乱,所以不现实了
    numberstyle         =   \ttfamily,    % 行号的样式,小五号,tt等宽字体
    showstringspaces    =   false,
    captionpos          =   t,      % 这段代码的名字所呈现的位置,t指的是top上面
    frame               =   lrtb,   % 显示边框
}

\makeatletter
\renewcommand{\section} {\@startsection{section}{1}{0mm}{-\baselineskip}{0.5\baselineskip}{\bf\leftline}}
\makeatother

\renewcommand{\algorithmicrequire}
{\textbf{Input:}}
\renewcommand{\algorithmicensure}
{\textbf{Output:}}

\makeatletter
\newenvironment{breakablealgorithm}
{% \begin{breakablealgorithm}
    \begin{center}
        \refstepcounter{algorithm}% New algorithm
        \hrule height.8pt depth0pt \kern2pt% \@fs@pre for \@fs@ruled
        \renewcommand{\caption}[2][\relax]{% Make a new \caption
            {\raggedright\textbf{\ALG@name~\thealgorithm} ##2\par}%
            \ifx\relax##1\relax % #1 is \relax
            \addcontentsline{loa}{algorithm}{\protect\numberline{\thealgorithm}##2}%
            \else % #1 is not \relax
            \addcontentsline{loa}{algorithm}{\protect\numberline{\thealgorithm}##1}%
            \fi
            \kern2pt\hrule\kern2pt
        }
    }{% \end{breakablealgorithm}
        \kern2pt\hrule\relax% \@fs@post for \@fs@ruled
    \end{center}
}
\makeatother

\begin{document}
    \maketitle
    
    \section{题目名称}

    这是一份简单的算法作业 \LaTeX 格式模板(该模板仅作为排版参考,不代表规范的答题格式)

    \subsection{解答}

    笔者使用的测试环境: Linux 系统, TeXLive 环境, TeX Studio 编辑器。

    编译命令: $\texttt{xelatex -synctex=1 -interaction=nonstopmode MyTemplate.tex}$。
    
    \quad \\
    
    \section{题目名称}

    题目描述。普通文本示例。数学环境示例 $n$, $G=(V, E)$。

    以下是一个有序列表。

    \begin{enumerate} 
        \item 有序列表条目(一)
        \item 有序列表条目(二)
        \item 有序列表条目(三)
    \end{enumerate}

    以下是一个无序列表。

    \begin{itemize} 
        \item 无序列表条目(一)
        \item 无序列表条目(二)
        \item 无序列表条目(三)
    \end{itemize}

    \subsection{算法设计}

    简要描述解决问题的算法的设计思路。

    \subsection{伪代码}

    \begin{breakablealgorithm}
        \renewcommand{\thealgorithm}{}
        \caption{伪代码示例}
        \begin{algorithmic}[1]
            \Require {问题的输入, 数据个数 $n$}
            \Ensure {问题的输出, 数组 $L$}
            \Function {func}{$a, b$}    % 定义函数, 第一个 {} 内是函数名,第二个 {} 内是形参
                \State \Return $a + b$  % 函数体
            \EndFunction
            % 这段伪代码没有什么实际的意义
            \State $count \gets 0$ % \State 表示一条语句, 除了 While/For/If 以外每条语句都必须以 \State 开头。
            \State $L$ $\gets$ $\text{Array}[1 \cdots n]$  % 伪代码中对数据结构没有严格的格式规定
            \While {$count < n$}    % While 循环
                \State $count \gets count + 1$ \Comment{这是注释}
                \State $L[count] \gets$ \Call{func}{$count, 1$} % \Call 函数调用
            \EndWhile
            \For {$i \gets 1 \cdots n$} % For 循环
                \If {$i \% 2 = 0$}  % 条件分支
                    \State $L[i] \gets 1$
                \ElsIf {$i \% 3 = 0$}   % Else If (可以没有)
                    \State $L[i] \gets 0$
                \Else                   % Else (可以没有)
                    \State $L[i] \gets i$
                \EndIf
            \EndFor
            \State \Return $L$ % 算法的输出
        \end{algorithmic}
    \end{breakablealgorithm}

    \subsection{时间复杂度}
    
    例: 该算法的时间复杂度为 $O(n)$。

    \quad \\    % 这里为了强行增加一行空行,防止上下内容连在一起。

    \section{另一道题}

    另一道题的题目描述。

    \subsection{算法设计}

    这道题的算法设计思路。

    \subsection{伪代码}

    \begin{breakablealgorithm}
        \renewcommand{\thealgorithm}{}
        \caption{伪代码另一个示例}
        \begin{algorithmic}[1]
            \Require {两个整数 $a, b$}
            \Ensure {输入整数的和}
            \State \Return $a+b$
        \end{algorithmic}
    \end{breakablealgorithm}

    \subsection{时间复杂度}

    该算法的时间复杂度为 $O(1)$。
    
    \quad \\
    
    
\end{document}